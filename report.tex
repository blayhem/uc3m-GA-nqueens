\documentclass[12pt]{article}
\usepackage[utf8]{inputenc}

\begin{document}

  \begin{titlepage}
    \begin{center}
        \vspace*{1cm}
        
        \Huge
        \textbf{Práctica 1 - N Reinas}

        \vspace{0.5cm}

        \huge
        Algoritmos Genéticos y Evolutivos
        
        
        \vspace{1.5cm}
        
        \large
        \textbf{Daniel Fernández Rico - 100303645}
        
        % \vfill
        \large
        Grado en Ingeniería Informática\\
        Curso 2017 - 2018\\

    \end{center}
  \end{titlepage}

  \begin{center}
    \Large
    \textbf{Title}
    
    \vspace{0.4cm}
    
  \end{center}

  - Mutacion:
  1. Primera aproximación: se mutan los dos hijos con probabilidad $p_m$. Aproximación clásica: por cada bit del individuo, se evalúa su probabilidad de mutación, y se cambia el bit (0-1). Al no ser codificación binaria, lo cambiamos por valores 0-N.

  2. Segunda aproximación: con (1) obtenemos resultados repetidos, nuestra codificación es una permutación de columnas (valores 0-N). Por lo tanto, evaluamos la probabilidad del individuo entero de mutar, y si es positiva hacemos un reordenamiento de sus valores (shuffle).

  3. Tercera aproximación: con (2) obtenemos valores demasiado aleatorios con una probabilidad de mutación elevada. Nuestra aproximación esta vez consiste de nuevo en evaluar la probabilidad de mutación bit a bit, y en caso de ser positiva, intercambiando dicho bit con otro bit del individuo de manera aleatoria. De esta forma, en caso de mutar 1 bit, el individuo sólo diferirá del original en 2 bits.

  Nota: la probabilidad de mutar depende de N, ya que el número de bits a mutar depende también del tamaño del individuo, así que para hacerlo más constante tomamos la probabilidad de 1/N y la multiplicamos por la probabilidad por individuo.

\end{document}
