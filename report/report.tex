\documentclass[12pt]{article}
\usepackage{makeidx}
\usepackage[margin=1in]{geometry}  % set the margins to 1in on all sides
\usepackage{graphicx}              % to include figures
\usepackage{amsmath}               % great math stuff
\usepackage{amsfonts}              % for blackboard bold, etc
\usepackage{amsthm}                % better theorem environments
\usepackage{makeidx}               % index
\usepackage[utf8]{inputenc}        % now we have tildes!
\usepackage{graphicx}
\usepackage{wrapfig}
\usepackage{listings}
\usepackage[spanish]{babel}
\usepackage{wrapfig}
\usepackage{enumerate}
\usepackage{hyperref}
\usepackage{titletoc}
\usepackage{tocbibind}
\usepackage{textcomp}
\usepackage{amsmath}
\usepackage{mathtools}
\usepackage{caption}
\usepackage{subcaption}
\usepackage{tikz}
\usepackage[T1]{fontenc}
\usepackage[many]{tcolorbox}
\usepackage{forloop}
\tcbuselibrary{listings}

% various theorems, numbered by section

\graphicspath{{static/}}

\begin{document}

\begin{titlepage}

\newcommand{\HRule}{\rule{\linewidth}{0.5mm}} % Defines a new command for the horizontal lines, change thickness here

\center % Center everything on the page

%----------------------------------------------------------------------------------------
% HEADING SECTIONS
%----------------------------------------------------------------------------------------

\textsc{\LARGE Universidad Carlos III de Madrid}\\[1cm] % Name of your university/college

%----------------------------------------------------------------------------------------
% LOGO SECTION
%----------------------------------------------------------------------------------------

\includegraphics[width=50mm]{uc3m_logo}\\[1cm] % Include a department/university logo - this will require the graphicx package

\textsc{\Large Algoritmos Genéticos y Evolutivos}\\[0.5cm] % Major heading such as course name
% \textsc{\large Grupo 84}\\[2cm] % Minor heading such as course title

%----------------------------------------------------------------------------------------
% TITLE SECTION
%----------------------------------------------------------------------------------------

\HRule \\[0.4cm]
{ \huge \bfseries Práctica 1: N Reinas.}\\[0.4cm] % Title of your document
\HRule \\[2cm]

%----------------------------------------------------------------------------------------
% AUTHOR SECTION
%----------------------------------------------------------------------------------------


% If you don't want a supervisor, uncomment the two lines below and remove the section above
\emph{Autor:}\\[0.7cm]

\begin{tabular}{rl}
    Daniel \textsc{Fernández}: &\texttt{daniel.f.rico@alumnos.uc3m.es}
\end{tabular}\\[2cm]


%----------------------------------------------------------------------------------------
% DATE SECTION
%----------------------------------------------------------------------------------------

{\large \today}\\[3cm] % Date, change the \today to a set date if you want to be precise
%----------------------------------------------------------------------------------------

\vfill % Fill the rest of the page with whitespace

\end{titlepage}

\tableofcontents

\newpage

\section{Problema 1}


\subsection{Interpretación de los datos recibidos}

- Mutacion:
1. Primera aproximación: se mutan los dos hijos con probabilidad $p_m$. Aproximación clásica: por cada bit del individuo, se evalúa su probabilidad de mutación, y se cambia el bit (0-1). Al no ser codificación binaria, lo cambiamos por valores 0-N.

2. Segunda aproximación: con (1) obtenemos resultados repetidos, nuestra codificación es una permutación de columnas (valores 0-N). Por lo tanto, evaluamos la probabilidad del individuo entero de mutar, y si es positiva hacemos un reordenamiento de sus valores (shuffle).

3. Tercera aproximación: con (2) obtenemos valores demasiado aleatorios con una probabilidad de mutación elevada. Nuestra aproximación esta vez consiste de nuevo en evaluar la probabilidad de mutación bit a bit, y en caso de ser positiva, intercambiando dicho bit con otro bit del individuo de manera aleatoria. De esta forma, en caso de mutar 1 bit, el individuo sólo diferirá del original en 2 bits.

Nota: la probabilidad de mutar depende de N, ya que el número de bits a mutar depende también del tamaño del individuo, así que para hacerlo más constante tomamos la probabilidad de 1/N y la multiplicamos por la probabilidad por individuo.


- Cruce:

\subsection{Preprocesado de los datos}

\subsection{Parametrización}

\begin{itemize}
    \setlength\itemsep{0em}
    \item Capas: \textbf{3}.
    \item Momentum: \textbf{0.2455}.
    \item Número de ciclos: \textbf{400}.
    \item El conjunto de datos, con un número reducido de patrones de
    entrada, requiere un modelo con una tasa de aprendizaje alta. Se fijó este
    valor en \textbf{0.61} y sin decaer durante el entrenamiento.
\end{itemize}

La elección de dichos parámetros se basó en una comparativa de los distintos
valores posibles, como podemos ver en la siguiente figura:

\begin{figure}[h]
    \center
    \includegraphics[width=\textwidth]{graphics}
    \caption{Evolución del acierto en la clasificación según los parámetros escogidos para el entrenamiento}
    \label{}
\end{figure}

\subsection{Resultados}


\newpage

\section{Problema 2}


\begin{figure}[h]
  \center
    \begin{subfigure}{.4\textwidth}
        \includegraphics[width=\textwidth]{cajero1_antecedentes}
        \caption{Cajero 1}
        \label{}
    \end{subfigure}
    \begin{subfigure}{.4\textwidth}
        \includegraphics[width=\textwidth]{cajero2_antecedentes}
        \caption{Cajero 2}
        \label{}
    \end{subfigure}
    \begin{subfigure}{.4\textwidth}
        \includegraphics[width=\textwidth]{cajero3_antecedentes}
        \caption{Cajero 3}
        \label{}
    \end{subfigure}
    \caption{Variación del error cuadrático medio al aumentar el número de
    antecedentes con 3 capas ocultas}
    \label{}
\end{figure}


\newpage

\section{Conclusiones, Problemas Encontrados y Opiniones Personales}

\section{Anexo}



\newcounter{atm}
\newcounter{prediction}

\forloop{atm}{1}{\value{atm} < 4}{
    \begin{figure}[h]
        \center
        \forloop{prediction}{1}{\value{prediction} < 11}{
            \begin{subfigure}{.35\textwidth}
                \includegraphics[width=\textwidth]{cajero\arabic{atm}_prediccion/cajero\arabic{atm}_prediccion-\arabic{prediction}}
                \caption{Predicción \arabic{prediction}}
                \label{}
            \end{subfigure}
        }
        \caption{Tendencias predichas dados los k valores anteriores para el cajero \arabic{atm}}
        \label{}
    \end{figure}
}

\end{document}
